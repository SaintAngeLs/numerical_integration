\documentclass[a4paper,12pt]{article}
\usepackage{amsfonts}
\usepackage{amssymb}
\usepackage{latexsym}
\usepackage{amsmath}
\usepackage{amsthm}
\usepackage{graphicx}
\usepackage{indentfirst}
\usepackage[polish]{babel}
\usepackage[T1]{fontenc}
\usepackage[cp1250]{inputenc}
\usepackage{setspace}
\usepackage{array}
\usepackage{multirow}
\usepackage{geometry}
\geometry{hdivide={2cm,*,2cm}}
\geometry{vdivide={2cm,*,2cm}}
\usepackage{titlesec}
\titlespacing{\section}{0ex}{1ex}{1ex}
\titleformat*{\section}{\sf\large\bfseries}
\titlespacing{\subsection}{0ex}{0.75ex}{0.75ex}
\titleformat*{\subsection}{\sf\bfseries}

\AtBeginDocument{
\addtolength{\abovedisplayskip}{-1ex}
\addtolength{\abovedisplayshortskip}{-1ex}
\addtolength{\belowdisplayskip}{-1ex}
\addtolength{\belowdisplayshortskip}{-1ex}
}
\newcolumntype{C}[1]{>{\centering\arraybackslash}m{#1}}
\newcommand{\razy}{\hspace{-0.5ex}\times\hspace{-0.5ex}}


\begin{document}

\def\tablename{Tabela} % bez tej linii nazw¹ tabeli by³aby "Tablica"


\noindent
\textbf{Pawe³ Keller, 999999, grupa 7, projekt 0, zadanie 00}


\section*{Wstêp}
Po krótkiej naradzie, razem z Wyk³adowc¹ zdecydowaliœmy siê udostêpniæ
Pañstwu niemal ca³y przyk³adowy plik raportu. Jedynie \emph{Wstêp} podmieniony
zosta³ na kilka informacji, które w³aœnie Pañstwo czytaj¹. Przypominam, ¿e wstêp
powinien zawieraæ opisan¹ w³asnymi s³owami i w trybie oznajmuj¹cym treœæ
zadania oraz krótk¹ wzmiankê o rezultatach przeprowadzonych
eksperymentów (taki raport ,,w pigu³ce'').

Proszê zwróciæ równie¿ uwagê, ¿e w rozdziale opisuj¹cym dan¹ metodê numeryczn¹
(proszê pamiêtaæ o odpowiedniej modyfikacji tytu³u tego rozdzia³u --- je¿eli ktoœ
odda raport z pozostawionym z tego przyk³adu tytu³em drugiego rozdzia³u,
to odejmê za to 3 punkty!) wszystkie u¿ywane symbole i pojêcia s¹
tam zdefiniowane. Innymi s³owy, nie wolno u¿ywaæ pojêæ lub symboli
wczeœniej niezdefiniowanych (najpóŸniej nale¿y definiowaæ symbol/pojêcie
w zdaniu nastêpuj¹cym po tym, w którym jest u¿yty po raz pierwszy).
Nie dotyczy to oczywiœcie pojêæ ogólnie znanych (np.\ \emph{ca³ka}).

Rozszerzony zosta³ te¿ \emph{Dodatek} o dowody dwóch teoretycznych faktów.
Nie jest to jednak przyk³ad jak \emph{Dodatek} (raport) koniecznie powinien wygl¹daæ,
a jedynie drobne uzupe³nienie materia³u z poprzedniego semestru --- jeœliby
kogoœ to zainteresowa³o --- oraz kilka dodatkowych, przyk³adowych wzorów
matematycznych. Dodano równie¿ bardzo wa¿n¹ \emph{Uwagê koñcow¹}.


\section*{Opis metody Newtona}
Za³ó¿my, ¿e $\alpha$ jest $r$-krotnym pierwiastkiem pewnej $r$-krotnie
ró¿niczkowalnej funkcji $f$, czyli takim, ¿e
$f(\alpha) = f'(\alpha) = \dots = f^{(r-1)}(\alpha)=0$
oraz $f^{(r)}(\alpha) \neq 0$. Wtedy dla przybli¿enia pocz¹tkowego
$x_0$ zmodyfikowana metoda Newtona wyznacza kolejne przybli¿enia $x_k$ ($k\geq 1$)
pierwiastka $\alpha$ zgodnie z nastêpuj¹cym wzorem:
\begin{equation}
\label{NewtonFormula}
  x_k = x_{k-1} - r \frac{f(x_{k-1})}{f'(x_{k-1})}.
\end{equation}
Je¿eli b³¹d $|e_0|$, gdzie $e_k = x_k - \alpha$ ($k \geq 0$), jest odpowiednio
ma³y, to
$$ % wzory bez numerów mo¿na te¿ uzyskaæ otoczeniem \begin{equation*}..\end{equation*}
  |e_k| \leq C |e_{k-1}|^2 \qquad (k \geq 1)
$$  
dla pewnej sta³ej $C$. Oznacza to, ¿e wystarczy ju¿ kilka iteracji, aby przybli¿yæ
pierwiastek $\alpha$ z kilkunastocyfrow¹ dok³adnoœci¹.

Poniewa¿ nie jest znana wystarczaj¹ca liczba iteracji potrzebna do wyznaczenia
pierwiastka z pewn¹ zadan¹ dok³adnoœci¹, w praktycznej realizacji metody Newtona
u¿ytej do wykonania eksperymentów opisanych w raporcie, iteracje \eqref{NewtonFormula}
wykonuje siê do momentu, a¿
\begin{equation}
\label{ErrEst}
  E_k := |x_k - x_{k-1}| \leq \delta
\end{equation}
dla pewnej zadanej tolerancji $\delta > 0$. Dodatkowo, obliczenia s¹ przerywane,
jeœli wartoœæ $k$ w~\eqref{NewtonFormula} robi siê nazbyt du¿a lub gdy $f(x_{k-1}) = 0$.
Ten ostatni warunek jest konieczny, poniewa¿ w wypadku zer wielokrotnych mamy
wtedy $f'(x_{k-1}) = 0$. W \emph{Dodatku} wykazano, ¿e w otoczeniu wielokrotnego
pierwiastka zachodzi nierównoœæ $|f(x_{k-1})| < |f'(x_{k-1})|$.


\section*{Eksperymenty numeryczne}
Zauwa¿my, ¿e gdy mamy obliczone trzy kolejne b³êdy przybli¿enia:
$e_{k-1}$, $e_{k}$ i $e_{k+1}$, to mo¿emy ³atwo szacowaæ wyk³adnik zbie¿noœci metody.
Przyjmuj¹c, ¿e $|e_{k+1}| \simeq C |e_{k}|^p$ oraz $|e_k| \simeq C |e_{k-1}|^p$,
to po zlogarytmowaniu powy¿szych dwóch przybli¿onych równoœci i odjêciu stronami,
dostaniemy
$$
  p \simeq \log\Big(\frac{|e_{k+1}|}{|e_k|}\Big)
           \log\Big(\frac{|e_{k}|}{|e_{k-1}|}\Big)^{-1}.
\vspace{1ex}
$$
W tabeli \ref{T1} przedstawiono wyniki obliczeñ dla kilku wybranych funkcji $f$ oraz
kilku wartoœci parametru $r$ w \eqref{NewtonFormula}. Podano oszacowan¹ wartoœæ
wyk³adnika zbie¿noœci $p$ (albo symbol ''\verb;?;'', gdy nie by³o mo¿liwe jego
oszacowanie --- kolejne trzy b³êdy nie mala³y lub nie by³y mniejsze ni¿ jeden).
Podano te¿ wartoœæ oszacowania \eqref{ErrEst} dla ostatniej ($k$-tej)
iteracji oraz faktyczn¹ wartoœæ b³êdu $|e_k|$.

Jak widaæ, przy odpowiednio dobranej wartoœci $r$ (równej krotnoœci pierwiastka)
otrzymujemy najszybsz¹ zbie¿noœæ, co najmniej rzêdu $2$. W wypadku dwóch ostatnich
przyk³adów z tabeli, przy optymalnym doborze parametru $r$, zbie¿noœæ jest a¿ szeœcienna,
co potwierdzono dowodem teoretycznym w \emph{Dodatku}. Na podstawie otrzymanych wyników widaæ
równie¿, ¿e z³e dobranie parametru $r$ mo¿e skoñczyæ siê w ma³o interesuj¹cy sposób.
Jeœli nie znamy zatem krotnoœci pierwiastka, to nale¿y przypisaæ $r:=1$.

Interesuj¹cy wydaje siê ostatni przyk³ad z tabeli \ref{T1}, dla którego nie uda³o
siê obliczyæ przybli¿enia pierwiastka z b³êdem mniejszym ni¿ $1.06\razy10^{-9}$ (algorytm
,,utkn¹³'' w tym punkcie). Opisany problem jest prawdopodobnie konsekwencj¹ faktu,
¿e obliczanie wartoœci wyra¿enia $\cos(x) - 1$ dla argumentów bliskich $0$ obarczone
jest bardzo du¿ym b³êdem zwi¹zanym ze spor¹ redukcj¹ cyfr znacz¹cych.

\begin{table}[!h]\vspace*{-2ex}
\caption{\footnotesize Oszacowana wartoœæ wyk³adnika zbie¿noœci $p$, oszacowanie $E_k$
b³êdu przybli¿enia, faktyczny b³¹d $|e_k|$ oraz numer iteracji $k$, w której spe³niony
by³ warunek $E_k < 10^{-12}$ dla kilku wybranych funkcji i kilku wartoœci~$r$
w \eqref{NewtonFormula}. Szukanym pierwiastkiem jest zawsze $\alpha = 0$,
a przybli¿enie pocz¹tkowe jest równe $x_0 = 0.5$. Jeœli w tabeli oszacowanie
b³êdu jest równe $0.0$, oznacza to, ¿e tyle wynosi³a wartoœæ $E_k$ lub
$f(x_{k-1})=0$ --- taka realizacja (a nie sprawdzanie, czy $f(x_k)=0$)
by³a prostsza w praktyce.}\vspace{-1.5ex}
\label{T1} % to niezbyt ³adna nazwa etykiety :(
\begin{center}
\begin{small}
\begin{tabular}{|C{20ex}|C{10ex}|C{10ex}|C{10ex}|C{10ex}|C{10ex}|}\hline
funkcja $f$ & $r$ & $p$ & $E_k$ & $|e_k|$ & $k$\\\hline
\multirow{3}{*}{$\exp(x)-1$}
& $1$ & $2.00$ & $1.1\razy10^{-16}$ & $5.4\razy10^{-17}$ & 5 \\
& $2$ & \verb;?; & --- & --- & $\infty$ \\
& $3$ & \verb;?; & --- & --- & $\infty$ \\\hline
\multirow{3}{*}{$(\exp(x)-1)^3$}
& $1$ & $1.00$ & $8.7\razy10^{-13}$ & $1.8\razy10^{-13}$ & 66 \\
& $2$ & $1.00$ & $7.9\razy10^{-13}$ & $3.9\razy10^{-13}$ & 26 \\
& $3$ & $2.00$ & $0.0$ & $1.1\razy10^{-16}$ & 6 \\
& $4$ & $1.00$ & $1.0\razy10^{-12}$ & $4.0\razy10^{-13}$ & 67 \\
& $5$ & \verb;?; & --- & --- & $\infty$ \\\hline
\multirow{2}{*}{$x^2(x-3)(x+1)$}
& $1$ & $1.00$ & $5.5\razy10^{-13}$ & $5.5\razy10^{-13}$ & 40 \\
& $2$ & $2.00$ & $3.2\razy10^{-16}$ & $0.0$ & 4 \\
& $3$ & $1.00$ & $6.9\razy10^{-13}$ & $2.3\razy10^{-13}$ & 41 \\
& $4$ & \verb;?; & --- & --- & $\infty$ \\\hline
\multirow{1}{*}{$\sin(x)^2$}
& $2$ & $3.00$ & $1.2\razy10^{-14}$ & $0.0$ & 4 \\\hline
\multirow{1}{*}{$\cos(x)-1$}
& $2$ & $3.01$ & $0.0$ & $1.1\razy10^{-9}$ & 4 \\\hline
\end{tabular}
\end{small}
\end{center}
\end{table}\vspace{-3ex}

Na rysunku \ref{R1} przedstawiono graficznie przedzia³y zbie¿noœci zmodyfikowanej
metody Newtona dla funkcji $(x+1)^2(x-1)^2(x-2)^2(x-4)^2$ i $r\in\{1,2,3,4\}$.
Jasne kolory oznaczaj¹ zbie¿noœæ (dla danego punktu pocz¹tkowego) do jednego
z czterech pierwiastków: $-1$, $1$, $2$ albo $4$. Otrzymane wyniki potwierdzaj¹
tezê, ¿e czasem bardzo istotny jest odpowiedni dobór przybli¿enia pocz¹tkowego, oraz
¿e z³e dobranie parametru $r$ mo¿e spowodowaæ brak zbie¿noœci metody.

\begin{figure}[h!]
\begin{center}
\includegraphics[width=8.5cm,height=1cm,angle=0]{fun3-1.png}\hspace*{-4ex}
\includegraphics[width=8.5cm,height=1cm,angle=0]{fun3-2.png}\\[1ex]
\includegraphics[width=8.5cm,height=1cm,angle=0]{fun3-3.png}\hspace*{-4ex}
\includegraphics[width=8.5cm,height=1cm,angle=0]{fun3-4.png}\vspace*{-4ex}
\end{center}
\caption{\footnotesize Przedzia³y zbie¿noœci zmodyfikowanej metody Newtona dla funkcji
$(x+1)^2(x-1)^2(x-2)^2(x-4)^2$. Kolorami czerwonym, zielonym, niebieskim oraz ¿ó³tym oznaczono
przybli¿enia pocz¹tkowe, które ,,prowadz¹'' -- odpowiednio -- do pierwiastków
$-1$, $1$, $2$ i $4$. Ciemny kolor oznacza brak zbie¿noœci. Kolejne rysunki (od lewego
górnego, wierszami) odpowiadaj¹ wartoœciom $1$, $2$, $3$ i $4$ parametru $r$
(porównaj wzór \eqref{NewtonFormula}).}
\label{R1}
\end{figure}


\section*{Dodatek}

Przedstawione na rysunku \ref{R1} przedzia³y zbie¿noœci doœæ czytelnie pokazuj¹, jak
dla wybranej funkcji $f$ zmiany parametru $r$ wp³ywaj¹ na zbie¿noœæ zmodyfikowanej
metody Newtona. Jednak od strony wizualnej mo¿na uznaæ te rysunki za ma³o ciekawe.

O wiele ciekawsze (wizualnie) obrazy otrzymaæ mo¿na, gdy zastosuje siê metodê Newtona
do wyznaczania pierwiastków funkcji zespolonej i odpowiednio pokoloruje obszary
zbie¿noœci metody. W dziedzinie zespolonej metoda Newtona nadal opisywana jest
wzorem \eqref{NewtonFormula}.

Na rysunku \ref{R2} przedstawiono obszary zbie¿noœci zmodyfikowanej metody Newtona
(z parametrem $r=2$) dla funkcji zespolonej
\begin{equation}
\label{FractFunct}
  f(z) = \big(1 + \frac{1}{4}\sin(2z) - \frac{1}{4}\cos(2z)\big)
         (z+1)^2(z-1)^2(z-2)^2(z-4)^2.
\end{equation}
Pierwiastkami tej funkcji s¹ liczby $-1$, $1$, $2$, $4$ i --- byæ mo¿e --- pewne
inne liczby zespolone. Na rysunku \ref{R2} pokolorowano wy³¹cznie te punkty z wybranego
prostok¹ta p³aszczyzny zespolonej, dla których metoda zbiega³a do jednego z czterech
podanych pierwiastków rzeczywistych (istotne s¹ jedynie aspekty wizualne obrazków),
pozosta³e punkty zaznaczono ciemno-szarym kolorem t³a.

\begin{figure}[h!]
\vspace*{-2ex}
\begin{center}
\includegraphics[width=8.5cm,height=6cm,angle=0]{fun2_2D2.png}\hspace*{-4ex}
\includegraphics[width=8.5cm,height=6cm,angle=0]{fun2_2D2_zoom.png}\\[-1.5ex]
\includegraphics[width=8.5cm,height=6cm,angle=0]{fun2_2D2_zoom_alt.png}\hspace*{-4ex}
\includegraphics[width=8.5cm,height=6cm,angle=0]{fun2_2D2_zoom_zoom.png}\vspace*{-5ex}
\end{center}
\caption{\footnotesize Obszary zbie¿noœci zmodyfikowanej ($r=2$) metody Newtona
dla funkcji \eqref{FractFunct}. Kolorami czerwonym, zielonym, niebieskim oraz ¿ó³tym
oznaczono przybli¿enia pocz¹tkowe, które ,,prowadz¹'' -- odpowiednio -- do pierwiastków
$-1$, $1$, $2$ i $4$. Ciemny kolor oznacza brak zbie¿noœci lub zbie¿noœæ do innego
pierwiastka. Kolejne rysunki odpowiadaj¹ ró¿nym rozwa¿anym obszarom.}
\label{R2}
\end{figure}

\subsection*{Uzasadnienia kilku faktów teoretycznych}

Na pocz¹tek poka¿emy, ¿e dla funkcji $f(x) = \sin(x)^2$ zmodyfikowana metoda Newtona
\eqref{NewtonFormula} z $r = 2$ jest zbie¿na szeœciennie do pierwiastka $\alpha = 0$,
dla przybli¿enia pocz¹tkowego odpowiednio bliskiego $\alpha$. £atwo zauwa¿yæ,
i¿ spe³nione s¹ równoœci
$$
  e_{k+1} = e_k - 2 \frac{f(x_k)}{f'(x_k)}
          = \frac{e_k f'(x_k) - 2 f(x_k)}{f'(x_k)}.
$$
Rozwijaj¹c $f(x_k)$ oraz $f'(x_k)$ w szereg Taylora oraz zauwa¿aj¹c, ¿e w rozwa¿anym
przyk³adzie $x_k = e_k$ i $f(0) = f'(0) = 0$, otrzymujemy
$$
  e_{k+1} = \frac{e_k \big(e_k f''(0) + \frac{1}{2} e_k^2 f'''(0)
                  + \frac{1}{6} e_k^3 f^{(4)}(\xi_0) \big)
                  - 2 \big( \frac{1}{2}e_k^2 f''(0) + \frac{1}{6} e_k^3 f'''(0)
                       + \frac{1}{24} e_k^4 f^{(4)}(\xi_1) \big)}
                 {e_k f''(\xi_2)}.
$$
dla pewnych wartoœci $\xi_0$, $\xi_1$ i $\xi_2$ le¿¹cych
pomiêdzy zerem a $x_k$. Poniewa¿ dla $f(x) = \sin(x)^2$ mamy $f'''(0) = 0$,
$f''(0) = 2$ oraz $f^{(4)}(0) = -8$, to jeœli przybli¿enie $x_k$ jest dostatecznie
bliskie~$0$, otrzymamy
$$
  |e_{k+1}| = \bigg|
              \frac{\frac{1}{6} e_k^4 f^{(4)}(\xi_0) - \frac{1}{12} e_k^4 f^{(4)}(\xi_1)}
                   {e_k f''(\xi_2)}
              \bigg|
            \simeq \frac{1}{3}|e_k|^3 .
\vspace{1ex}
$$

W wypadku funkcji $f(x) = \cos(x) - 1$ dowód, ¿e zmodyfikowana metoda Newtona
zbiega do pierwiastka $\alpha = 0$ z wyk³adnikiem zbie¿noœci równym $3$ jest analogiczny.

Na koniec uzasadnimy, ¿e w pewnym otoczeniu wielokrotnego zera $\alpha$ odpowiednio
wiele razy ró¿niczkowalnej funkcji $f$ zachodzi warunek $|f(x)| < |f'(x)|$. W dowodzie
za³o¿ymy dla uproszczenia, ¿e krotnoœæ zera wynosi $2$. Dodatkowo wymagamy,
aby $|f''(\alpha)| < \infty$. W takim wypadku dla $x$ bliskich $\alpha$, pamiêtaj¹c
¿e $f(\alpha)=f'(\alpha)=0$, otrzymujemy
$$
  f(x) = f(\alpha) + (x-\alpha) f'(\alpha) + \frac{(x-\alpha)^2}{2} f''(\xi_0)
       = \frac{(x-\alpha)^2}{2} f''(\xi_0)
$$
oraz
$$
  f'(x) = f'(\alpha) + (x-\alpha) f''(\xi_1) = (x-\alpha) f''(\xi_1)
\vspace{0.5ex}
$$
dla pewnych $\xi_0$ oraz $\xi_1$ le¿¹cych pomiêdzy $\alpha$ i $x$. Poniewa¿
$0\! <\! |f''(\alpha)|\! <\! \infty$, to dla $x$ odpowiednio bliskich $\alpha$ zachodzi
$$
  \frac{1}{2} \big| (x-\alpha)^2 f''(\xi_0) \big| <
  \big| (x-\alpha) f''(\xi_1) \big|,
$$
co koñczy dowód.

\medskip


\section*{Uwaga koñcowa (bardzo wa¿na)}
Absolutnie zabronione jest kopiowanie zdañ (sformu³owañ) z tego raportu.
Niezastosowanie siê to powy¿szego zastrze¿enia mo¿e byæ potraktowane jako plagiat.
Najbezpieczniej oprzeæ w³asny raport na do³¹czonym \emph{szkielecie raportu}, a ten plik
traktowaæ jedynie jako wskazówki dotycz¹ce pisania tekstów w \LaTeX-u.


\end{document}
